\documentclass[]{article}

\usepackage{paracol,lipsum}
\usepackage{verse}
\usepackage[margin=.2in]{geometry}
\usepackage{tasks}
\usepackage{exsheets}
\SetupExSheets[question]{type=exam}

\usepackage{amssymb,latexsym}
\usepackage{amsmath,amsbsy,bbm}
\usepackage{epsfig,bm,color}
\usepackage{cjhebrew}
\usepackage{nicefrac}
\usepackage{graphicx,comment}
\usepackage{slashed}
%\usepackage{hyperref}
\unitlength=1mm

\usepackage{dutchcal}

\usepackage{calligra}

\DeclareMathAlphabet{\mathcalligra}{T1}{calligra}{m}{n}
\DeclareFontShape{T1}{calligra}{m}{n}{<->s*[2.2]callig15}{}
\newcommand{\scriptr}{\mathcalligra{r}\,}
\newcommand{\boldscriptr}{\pmb{\mathcalligra{r}}\,}

\def\eqref#1{{(\ref{#1})}}

\newcommand{\caf}{\text{\cjRL{b}}}
\newcommand{\he}{${}^4$He}
\newcommand{\hes}{${}^3$He}
\newcommand{\tr}{${}^3$H}
\newcommand{\ls}{\ve{L}\cdot\ve{S}}
\newcommand{\eps}{\epsilon}
\newcommand{\as}{a_s}
\newcommand{\at}{a_t}
\newcommand{\ecm}{E_\textrm{\small c.m.}}
\newcommand{\dq}{\mbox{d\hspace{-.55em}$^-$}}
\newcommand{\mpis}{$m_\pi=137~${\small MeV}}
\newcommand{\mpim}{$m_\pi=450~${\small MeV}}
\newcommand{\mpil}{$m_\pi=806~${\small MeV}}
\newcommand{\muh}{\mu_{^3\text{\scriptsize He}}}
\newcommand{\mut}{\mu_{^3\text{\scriptsize H}}}
\newcommand{\mud}{\mu_\text{\scriptsize D}}
\newcommand{\pode}{\beta_{\text{\scriptsize D},\pm1}}
\newcommand{\poh}{\beta_{^3\text{\scriptsize He}}}
\newcommand{\pot}{\beta_{^3\text{\scriptsize H}}}
\newcommand{\com}[1]{{\scriptsize \sffamily \bfseries \color{red}{#1}}}
\newcommand{\eg}{\textit{e.g.}\;}
\newcommand{\ie}{\textit{i.e.}\;}
\newcommand{\cf}{\textit{c.f.}\;}
\newcommand{\be}{\begin{equation}}
\newcommand{\ee}{\end{equation}}
\newcommand{\la}{\label}
\newcommand{\ber}{\begin{eqnarray}}
\newcommand{\eer}{\end{eqnarray}}
\newcommand{\nn}{\nonumber}
\newcommand{\half}{\frac{1}{2}}
\newcommand{\thalf}{\nicefrac[]{3}{2}}
\newcommand{\bs}[1]{\ensuremath{\boldsymbol{#1}}}
\newcommand{\bea}{\begin{eqnarray}}
\newcommand{\eea}{\end{eqnarray}}
\newcommand{\beq}{\begin{align}}
\newcommand{\eeq}{\end{align}}
\newcommand{\bk}{\bs k}
\newcommand{\bt}{B_{^{3}\text{H}}}
\newcommand{\bh}{B_{^{3}\text{He}}}
\newcommand{\bd}{B_\text{D}}
\newcommand{\ba}{B_\alpha}
\newcommand{\rgm}{$\mathbb{R}$GM}
\newcommand{\ev}[1] {|\bra #1  \ket |^2}
\newcommand{\lam}[1]{$\Lambda=#1~$fm$^{-1}$}
\newcommand{\parg}[1] {\paragraph*{-\,\textit{#1}\,-}}
\newcommand{\nopi}{\pi\hspace{-6pt}/}
\newcommand{\ve}[1]{\ensuremath{\boldsymbol{#1}}}
\newcommand{\xvec}{\bs{x}}
\newcommand{\rvec}{\bs{r}}
\newcommand{\sgve}{\ensuremath{\boldsymbol{\sigma}}}
\newcommand{\tave}{\ensuremath{\boldsymbol{\tau}}}
\newcommand{\na}{\nabla}
\newcommand{\bra}{\langle}
\newcommand{\ket}{\rangle}
\newcommand{\tx}{\tilde{x}}
\newcommand{\eftnopi}{\mbox{EFT($\slashed{\pi}$)}}
\newcommand{\red}[1]{\textcolor{red}{#1}}
\newcommand{\green}[1]{\textcolor{green}{#1}}
\newcommand{\blue}[1]{\textcolor{blue}{#1}}
\newcommand{\drei}[1]{\delta^{(3)}\!\left(#1\right)}
\newcommand{\ddrei}[1]{\delta_{\tiny \Lambda}^{(3)}\!\left(#1\right)}

\newcommand{\ecce}[2]{\paragraph*{Ecce: #1}\texttt{\textcolor{blue}{#2}}}

\newcommand{\tqt}[1]{\begin{flushleft}---\textquotedblleft\textit{#1}\textquotedblright\par\end{flushleft}}
\newcommand{\attrib}[1]{%
\nopagebreak{\raggedleft\footnotesize #1\par}}
\renewcommand{\poemtitlefont}{\normalfont\large\itshape\centering\bf}

\begin{document}
\columnratio{0.5}
\begin{paracol}{2}

\subsection*{\small \today}
\tqt{What I tell my niece.}
The mutual attraction between two or more objects does not guarantees that they ``end up'' --
which means after a sufficiently long time during which all transients of the initial encounter
equilibrated -- in a stable, localized state. Their paths through the world may still diverge and
never overlap despite affectionate meetings for some finite intervals.

Now, can {\it you} predict whether or not this behaviour pertains to any number of objects of this
kind? To give you some examples, although two planets attract each other gravitationally, they might
pass each other and end up in different corners of our universe. Is that scenario possible also for
the entirety of all planets, stars, point-like-on-astronomical-scales masses, \ie, is Newtonian
gravity consistent with an ever expanding universe, or does it demand its eventual collapse?
Secondly, a man and a women meet late at night in the swimming pool of a hotel in Manchester,
the rain knocks steadily against the glass roof, the blue tiles reflect only few rays of light, and
the atmosphere conveys a feeling of security in light of a harsh nature. The two talk, they experienced
similar things up to this point, they were both not born in England and communicate with each other not
in their mother tongue.
\switchcolumn

\subsection*{\small What I tell a machine.}
\begin{gather}
m_i\ddot{\ve{x}}_i=
\end{gather}
\begin{question}
Does an unbounded trajectory exist for any number of particles whose
mutual interaction does allow for such an unstable state?
\begin{tasks}
\task Put the system in a ``Fermi'' box.
\task Put that box into a gravitational background field whose centre is identical with that of the box.
\task Let the particles be connected to each other via springs,\ie , assume a
 harmonic-oscillator interaction.
\end{tasks}
\end{question}
\end{paracol}

\end{document}